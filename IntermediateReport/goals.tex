The project can be split into two parts. The goal of the first part is to build a GNN that is capable of assigning a 2- or 3-dimensional lattice its Bravais class. 
For this instance, different lattices need to be created programmatically. Furthermore, they have to be labeled so that they can serve as training data.
Once we have a good amount of training data at hand, we need to determine the optimal structure of the GNN. This amounts to finding the
right functions $\phi_{upd},\phi_{mes},\phi_{agg}$ in equation~\ref{eq:mess_pass} and to determine which edge-/node-features do best.
In the second part of the project, we will build upon the results in the first part and take a look at more complex structures. The goal is to distinguish between percolating and non-percolating structures.