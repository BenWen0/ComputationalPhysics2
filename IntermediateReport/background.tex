This introductory section lays the theoretical foundation of Bravais lattices and GNNs. 
We start with a short recap of lattice structures in 2d and 3d and then continue with 
the most fundamental background of Graph-Neural Networks (GNN). 
Results on Bravais lattices can be found in many textbooks. Subsection~\ref{sec:brav_latt} follows~\cite{symGroupsApplications}.
Introduction to Graph Neural Networks can be found for example in~\cite{IntroMessagePassing} which is also the main reference for section~\ref{sec:fund_gnns}.

\subsection{Bravais lattice}
\label{sec:brav_latt}
Let $d\in\N$ and $\{b_i\}_{i=1,\dots,d}\subset \R^d$ a basis of $\R^d$. The set 
\begin{equation*}
    \Omega\coloneq\left\{\sum_{i=1}^{d}z_i b_i: z_i\in\Z\,\forall i\in\{1,\dots,d\}\right\}
\end{equation*}
is called a $d$-dimensional lattice. Given any subset $S\subset\R^d$, we define its point group $G_S$ to be
\begin{equation*}
    G_S\coloneq \left\{M\in O(d) : MS=S\right\}\subset O(d).
\end{equation*}
$G_S$ is obviously a subgroup of $O(d)$. We say that two $d$-dimensional lattices $\Omega_1, \Omega_2\subset \R^d$ are of the same Bravais type if
there exists $g\in GL(n,\Z)$ such that $G_{\Omega_1}=g G_{\Omega_2} g^{-1}$. 
Being of the same Bravais type introduces an equivalence relation on the set of all $d$-dimensional lattices. We refer to the equivalence classes as Bravais classes.
A natural question arising is about the total number of Bravais classes. Despite being a very interesting and challenging problem,
we will leave this question to the mathematicians. For us, the result is more important than the actual proof. One obtains the following result:
For $d=2$ there are 5 Bravais classes and for $d=3$ there are 14 Bravais classes. 
They can be distinguished by the relative lengths of the basis vectors $b_i$ and the angles between them.

\subsection{Fundamentals of GNNs}
\label{sec:fund_gnns}
To talk about graphs, we first have to agree on some notation: Let $V$ be a set and $E\subset V\times V$. The tuple $G=(V,E)$ is called a graph. Furthermore, we call an element $x\in V$ a node and a tuple 
$(x, y)\in E$ a directed edge from $x$ to $y$, and we say that $x$ is a neighbor of $y$. 
In case $(x,y)\in E$ implies $(y,x)\in E$, we speak of an undirected graph.
In that case, we can think of elements in $E$ as unordered tuples $\{x,y\}$ instead of ordered ones. We still call $\{x,y\}$ an edge between $x$ and $y$.
For $y\in V$ we define the neighborhood $N_y$ of $y$ to be the set of all neighbors, i.e.
\begin{equation}
    \label{eq:def_neighbors}
    N_y\coloneq\left\{x\in V : (x,y)\in E\right\}\subset V.
\end{equation}
Furthermore, we assign to each node $x\in V$ a vector $v_x\in\R^n$ called node feature and to each edge $(x, y)\in E$ a vector $e_{x,y}\in\R^m$ called edge feature. 

Roughly, a GNN takes a graph with all its nodes and edge features as an input and manipulates these features in each step. 
More than that, a GNN can transform the structure of the graph itself, e.g. by introducing new nodes or edges. 
However, we will not go into detail about this possibility and stick to the simpler case of manipulating only node and edge-features.
Furthermore, we restrict ourselves to the case where the GNN does not alter the edge features.
\begin{SCfigure}[1][h]
\centering
\label{fig:gnn}
\begin{tikzpicture}[
    node_r/.style={circle, draw=red!60, fill=red!5, very thick, minimum size=7mm},
    node_o/.style={circle, draw=orange!60, fill=orange!5, very thick, minimum size=7mm},
    node_g/.style={circle, draw=gray!60, fill=gray!5, very thick, minimum size=7mm},
    node/.style={circle, draw=green!60, fill=green!5, very thick, minimum size=7mm},
    mess/.style={rectangle, draw=green!60, fill=green!5, very thick, minimum size=7mm},
    ]
    %Nodes step 1
    \node[node_r] (centernode)                          {$v_c$};
    \node[node] (top)           [above=of centernode]   {$v_x$};
    \node[node_o] (bottom)        [below=of centernode]   {};
    \node[node] (right)         [right=of centernode]   {$v_y$};
    \node[node] (left)          [left=of centernode]    {$v_z$};
    \node[node_g] (llt)           [above left=of left]    {};
    \node[node_g] (llb)           [below left=of left]    {};
    \node[node_g] (rrt)           [above right=of right]    {};
    \node[node_g] (rrb)           [below right=of right]    {};

    %Nodes step 2
    \node[mess] (mess_top)           [below=of bottom]   {$m_x$};
    \node[node_r] (centernode_2)  [below=of mess_top]     {$v_c$};
    \node[mess] (mess_right)         [right=of centernode_2]   {$m_y$};
    \node[mess] (mess_left)          [left=of centernode_2]    {$m_z$};

    %Nodes step 3
    \node[node_r] (centernode_3)  [below=of centernode_2]     {$v_c$};
    \node[mess] (mess_final)      [left=of centernode_3]   {$m_{x+y+z}$};

    %Nodes step 4
    \node[node_r] (centernode_4)  [below=of centernode_3]     {$\widehat{v_c}$};
    
    %Lines step 1
    \draw[->, thick]  (top.south) -- node[anchor=west]{$e_{x,c}$} (centernode.north);
    \draw[->, thick] (left.east) -- node[anchor=south]{$e_{z,c}$} (centernode.west);
    \draw[->, thick] (right.west) -- node[anchor=north]{$e_{y,c}$} (centernode.east);
    \draw[->, thick, gray] (centernode.south) -- (bottom.north);
    \draw[->, thick, gray] (left.north west) -- (llt.south east);
    \draw[->, thick, gray] (left.south west) -- (llb.north east);
    \draw[->, thick, gray] (right.north east) -- (rrt.south west);
    \draw[->, thick, gray] (right.south east) -- (rrb.north west);

    %Lines step 2
    \draw[->, thick]  (mess_top.south) -- (centernode_2.north);
    \draw[->, thick] (mess_left.east) -- (centernode_2.west);
    \draw[->, thick] (mess_right.west) -- (centernode_2.east);

    %Lines step 3
    \draw[->, thick]  (mess_final.east) -- (centernode_3.west);

    \draw[->, very thick, gray] (-4,-0) -- node[anchor=east, black]{$\phi_{mes}$} (-4,-5.2);
    \draw[->, very thick, gray] (-4,-5.5) -- node[anchor=east, black]{$\phi_{agg}$} (-4,-7.2);
    \draw[->, very thick, gray] (-4,-7.5) -- node[anchor=east, black]{$\phi_{upd}$} (-4,-9.2);
    %\draw[->] (rightsquare.south) .. controls +(down:7mm) and +(right:7mm) .. (lowercircle.east);
\end{tikzpicture}
\caption{Illustration of the message passing procedure according to equation~\eqref{eq:mess_pass}. 
The neighbors of node $c$ (red) are the nodes $x,y,z$ (green). Attention: According to equation~\eqref{eq:def_neighbors} the orange node is not regarded as a neighbor of $c$ as the edge points in the wrong direction.
For each neighbor ($x,y,z$), $\phi_{mes}$ computes messages ($m_x,m_y,m_z$) which are sent to $c$.
Then, $\phi_{agg}$ aggregates these three messages and outputs one overall message $m_{x+y+z}$ that is sent to $c$. 
In the last step, $\phi_{upd}$ updates the node value $v_c$ to a new value $\widehat{v_c}$.  
}
\end{SCfigure}
Let us make these ideas a bit more rigorous (see figure~\ref{fig:gnn} for an illustration):  
Let $(V,E)$ be a graph with node features $\{v_x\in\R^n : x\in V\}$ and edge features $\{e_{x,y}\in\R^m:(x,y)\in E\}$.
For each $c\in V$ a new node feature $\widehat{v_c}$ is calculated according to the following rule
\begin{equation}
    \label{eq:mess_pass}
    \widehat{v_c}=\phi_{upd}\left(
        v_c, \phi_{agg}\left(\left\{
            \phi_{mes}\left(v_y,v_c, e_{y,c}\right):y\in N_c
            \right\}\right)
    \right),
\end{equation}
where $\phi_{upd},\phi_{agg},\phi_{mes}$ denote differentiable functions. These three functions are commonly interpreted as follows:
$\phi_{mes}$ takes as inputs the node value $v_c$ and the node value $v_y$ of one neighbor $y$ of $c$ as well as the value $e_{y,c}$ of the edge $(y,c)$. 
Depending on these inputs, it then computes a value, which can be thought of as a message originating from node $y$ which is sent to node $c$.
$\phi_{agg}$ collects all these messages to node $c$ and aggregates them in some way, so that the output can be thought of as one overall message to node $c$.
$\phi_{upd}$ takes this overall message as well as the value of node $c$ and computes, how the value of node $c$ is altered.
Unsurprisingly, this scheme is called Message-Passing-Layer. Given a specific problem, that is required to be solved by a GNN, the challenge is to make appropriate choices
for $\phi_{upd},\phi_{agg},\phi_{mes}$ that suit the problem at hand. Furthermore, one has to decide how many iterations of the above procedure are suitable.