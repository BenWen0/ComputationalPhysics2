\vspace{-0.7cm}
\textbf{Observation:} GNN can solve percolation problem $\implies$ GNN can solve connection problem\par
\textbf{Question:} Can a GNN solve the connection problem?\par
\textbf{Answer:} No!\par
Assumption: there is a GNN with \textbf{2} layers capable of solving the connection problem\par
\begin{figure}
    \centering
    \begin{tikzpicture}[
        node/.style={circle, inner sep=0pt, fill=black, minimum size=3mm},
        node_missing/.style={circle, inner sep=0pt, fill=black, minimum size=1mm},
        node_r/.style={circle, inner sep=0pt, fill=red, minimum size=3mm}
    ]
    \node[node_r, label={$0$}] (n1) [] {};
    \node[node, label={$1$}] (n2) [right=of n1] {};
    \node[node, label={$2$}] (n3) [right=of n2] {};
    \node[node, label={$3$}] (n4) [right=of n3] {};
    \node[node, label={$4$}] (n5) [right=of n4] {};
    \node[node_r, label={$5$}] (n6) [right=of n5] {};
    \draw[very thick]  (n1) -- (n2);
    \draw[very thick]  (n2) -- (n3);
    \draw[very thick]  (n3) -- (n4);
    \draw[very thick]  (n4) -- (n5);
    \draw[very thick]  (n5) -- (n6);
    \end{tikzpicture}
\end{figure}
After one message passing steps:
\begin{figure}
    \centering
    \begin{tikzpicture}[
        node/.style={circle, inner sep=0pt, fill=black, minimum size=3mm},
        node_missing/.style={circle, inner sep=0pt, fill=black, minimum size=1mm},
        node_r/.style={circle, inner sep=0pt, fill=red, minimum size=3mm}
    ]
    \node[node_r, label={$0$}] (n1) [] {};
    \node[node, label={$1$}] (n2) [right=of n1] {};
    \node[node, label={$2$}] (n3) [right=of n2] {};
    \node[node, label={$3$}] (n4) [right=of n3] {};
    \node[node, label={$4$}] (n5) [right=of n4] {};
    \node[node_r, label={$5$}] (n6) [right=of n5] {};
    \draw[very thick]  (n1) -- (n2);
    \draw[very thick]  (n2) -- (n3);
    \draw[very thick]  (n3) -- (n4);
    \draw[very thick]  (n4) -- (n5);
    \draw[very thick]  (n5) -- (n6);
    \draw [->, very thick, green] (n1) to [out=45,in=135] (n2);
    \draw [->, very thick, green] (n6) to [out=135,in=45] (n5);
    \end{tikzpicture}
\end{figure}
After two message passing step:
\begin{figure}
    \centering
    \begin{tikzpicture}[
        node/.style={circle, inner sep=0pt, fill=black, minimum size=3mm},
        node_missing/.style={circle, inner sep=0pt, fill=black, minimum size=1mm},
        node_r/.style={circle, inner sep=0pt, fill=red, minimum size=3mm}
    ]
    \node[node_r, label={$0$}] (n1) [] {};
    \node[node, label={$1$}] (n2) [right=of n1] {};
    \node[node, label={$2$}] (n3) [right=of n2] {};
    \node[node, label={$3$}] (n4) [right=of n3] {};
    \node[node, label={$4$}] (n5) [right=of n4] {};
    \node[node_r, label={$5$}] (n6) [right=of n5] {};
    \draw[very thick]  (n1) -- (n2);
    \draw[very thick]  (n2) -- (n3);
    \draw[very thick]  (n3) -- (n4);
    \draw[very thick]  (n4) -- (n5);
    \draw[very thick]  (n5) -- (n6);
    \draw [->, very thick, green] (n1) to [out=45,in=135] (n2);
    \draw [->, very thick, green] (n2) to [out=45,in=135] (n3);
    \draw [->, very thick, green] (n6) to [out=135,in=45] (n5);
    \draw [->, very thick, green] (n5) to [out=135,in=45] (n4);
    \end{tikzpicture}
\end{figure}