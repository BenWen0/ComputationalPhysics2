\label{sec:creationLattice}

To be precise, we will not create lattice according to the definition given in sectio~\ref{sec:brav_latt}, 
as this would require the creation of sets with infinite elements. Clearly, a GNN can only deal with finitely many nodes.
Hence, we will only create finite subsets of lattice but, for sake of simplicity, we will still call them lattices.

Let us start with the creation of two-dimensional lattices.
Choose $a,b\in\R_+$, $\phi\in(0,\pi)$ and set
$e_x=(0,a)^T,\,e_y=(b\cos(\phi), b\sin(\phi))^T$ (for visulaization ogf $a,b$ and $\theta$, see figure~\ref{fig:bravais2D}). 
Then, the set $\tilde{\Omega}=\Z e_x\times\Z e_y\subset\R^2$ is a two-dimensional lattice.
As mentioned, we only work with finite subsets of $\Omega$. Hence, we choose $N_x,N_y\in\N$ and set $\Omega=\N_{\leq N_x}e_x\times\N_{\leq N_y}e_y\subset\tilde{\Omega}$.
This lattice now has $N_xN_y$ elements.
Next, we add some noise to the elements in $\Omega$ and restrict their coordinates to a certain range. For this instance, let $\mu,\sigma,s\in\R_+$ and draw random samples from
a normal distribution with mean value $\mu$ and standard deviation $\sigma$. Secondly, we scale these random numbers by the factor $s$ 
and add the scaled noise to all elements in $\Omega$ (component-wise, i.e. add noise to the first coordinate of the elements in $\Omega$ as well as to the second coordinate).
Restricting all coordinates to a certain range is simply done. Let $x_{max}, y_{max}\in\R_+$ and replace each $(x,y)\in\Omega$ with $(x \mod x_{max}, y \mod y_{max})$ 
\comment{modular arithmetic with non-integers values: stupid!}.
Next, we want to turn our lattice $\Omega$ into a graph $(V,E)$. Obviously, we can set $V=\Omega$ and what remains is the choice of edges. For this, we take $r\in\R_+$ and set
\begin{equation*}
    E=\{(v, w)\in V\times V: \norm{v-w}<r\},
\end{equation*}
where $\norm{\cdot}$ denotes the standard norm in $R^d$ \comment{not quite, actually I used the 3-norm}.
Lastly, we choose $p_n,p_e\in[0,1]$ and randomly delete nodes with probability $p_n$ (and edges with probability $p_e$ respectively).

The creation of three-dimensional lattices is completely analogously. The only difference is that we have to pick three basis vectors $e_x,e_y,e_z$ which have length $a,b,c\in\R_+$ and 
enclose angles $\phi=\angle(e_x, e_y),\,\psi=\angle(e_x,e_z),\,\chi=\angle(e_y,e_z)$. 

\comment{specfic settings for $a,b,..$, how many graphs per bravais class, ...}

Figure~\ref{fig:examplesLattices} shows two different lattices created by the above procedure.

\begin{figure}
    \centering
    \begin{subfigure}[t]{0.4\textwidth}
        \centering
        \caption{2d}
    \end{subfigure}
    \hfill
    \begin{subfigure}[t]{0.4\textwidth}
        \centering
        \caption{3d}
    \end{subfigure}
    \caption{Examples of lattices created by the procedure described in section~\ref{sec:creationLattice}. \comment{TODO}}
    \label{fig:examplesLattices}
\end{figure}

\comment{Furthermore, they have to be labeled so that they can serve as training data.}