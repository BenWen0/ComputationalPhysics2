\label{sec:creationLattice}

To be precise, we will not create lattices according to the definition given in section~\ref{sec:brav_latt}, 
as this would require creating sets with infinitely many elements. Clearly, a GNN can only deal with finitely many nodes.
Hence, we will only create finite subsets of lattice but, for sake of simplicity, we will still call them lattices.

Let us start with the creation of two-dimensional lattices.
Choose $a,b\in\R_+$, $\phi\in(0,\pi)$ and set
$e_x=(0,a)^T$ as well as $e_y=(b\cos(\phi), b\sin(\phi))^T$ (see figure~\ref{fig:bravais2D} for a visualization of $a,b$ and $\theta$). 
Then, the set $\tilde{\Omega}=\Z e_x\times\Z e_y\subset\R^2$ is a two-dimensional lattice.
As mentioned, we only work with finite subsets of $\tilde{\Omega}$. Hence, we choose $N_x,N_y\in\N$ and set $\Omega=\N_{\leq N_x}e_x\times\N_{\leq N_y}e_y\subset\tilde{\Omega}$.
Now, this lattice has $N_xN_y$ elements.
Next, we add some noise to the elements in $\Omega$ and restrict their coordinates to a certain range. For this instance, let $\mu,\sigma,s\in\R_+$. Firstly, draw random samples from
a normal distribution with mean value $\mu$ and standard deviation $\sigma$. Secondly, we scale these random numbers by the factor $s$ 
and add the scaled noise to all elements in $\Omega$ (coordinate-wise, i.e. add noise to the first coordinate of the elements in $\Omega$ as well as to the second coordinate).
Restricting all coordinates to a certain range is easily done. Let $x_{max}, y_{max}\in\R_+$ and replace each $(x,y)\in\Omega$ with $(x \mod x_{max}, y \mod y_{max})$ 
Next, we want to turn our lattice $\Omega$ into a graph $(V,E)$. Obviously, we can set $V=\Omega$ and what remains is the choice of edges. For this, we take $r\in\R_+$ and set
\begin{equation*}
    E=\{(v, w)\in V\times V: \norm{v-w}<r\},
\end{equation*}
where $\norm{\cdot}$ denotes the standard norm in $R^d$, i.e. nodes that are close together will be connected.
As $\norm{v-w}=\norm{w-v}$, the graph $(V,E)$ is undirected.
Lastly, we choose $p_n,p_e\in[0,1]$ and randomly delete nodes with probability $p_n$ (and edges with probability $p_e$ respectively).
This step finishes the creation of 2d lattices with artificial artifacts.
The creation of three-dimensional lattices goes analogously. The only difference is that we have to pick three basis vectors $e_x,e_y,e_z$ which have length $a,b,c\in\R_+$ and 
enclose angles $\phi=\angle(e_x, e_y),\,\psi=\angle(e_x,e_z),\,\chi=\angle(e_y,e_z)$. 
As an example, figure~\ref{fig:examplesLattices} shows two different lattices created by this scheme.

\begin{figure}
    \centering
    \begin{subfigure}[t]{0.45\textwidth}
        \centering
        \includegraphics[width=\textwidth]{Background_Ch1/square2d.png}
        \caption{2d}
    \end{subfigure}
    \hfill
    \begin{subfigure}[t]{0.45\textwidth}
        \centering
        \includegraphics[width=\textwidth]{Background_Ch1/cubic3d.png}
        \caption{3d}
    \end{subfigure}
    \caption{Examples of lattices created by the procedure described in section~\ref{sec:creationLattice}.}
    \label{fig:examplesLattices}
\end{figure}

According to the above procedure, we are able to generate training and test datasets for our GNN, both in 2d and in 3d.
We start with a description of the 2d-dataset. We created graphs with 100 nodes ($N_x=N_y=10$).
Recall that depending on the specific values of $a,b$ and $\theta$, different graphs for different Bravais classes can be generated 
(take again a look at figure~\ref{fig:bravais2D} as a reminder, which values for $a,b$ and $\theta$ result in which Bravais class).
Furthermore, recall that some Bravais classes require $\theta=\frac{\pi}{2}$. In order to introduce further artificial noise, we allow for values
in the range $(0.99\frac{\pi}{2}, 1.01\frac{\pi}{2})$.
Accordingly, when the side lengths are required to fulfill $a=b$, we allow for deviations of $1\%$, 
i.e. such that $0.99\leq\frac{a}{b}\leq1.01$.
Furthermore, we set the minimal side length to be 0.1 and the maximal side length to 10. 
The parameters $\sigma,\mu$ and $s$ determining the above-mentioned Gaussian noise were chosen to be $\sigma=0.5$ and $\mu=0$. The amplitude $s$ of
the noise was taken to be $s=0.07a$ in x-direction (depending on the side-length $a$) as well as $s=0.07b$
in y-direction. Next, we restricted all coordinates to the range $(x_{max},y_{max})=(N_xa,N_yb)=(10a,10b)$.
The probabilities $p_n,p_e$ for randomly dropping nodes/edges were taken to be $p_n=p_e=0.01$.
According to these settings, we created 10000 graphs per Bravais class. Since there are 5 Bravais classes in total, the
2d-dataset consists of 50000 graphs, $20\%$ were taken to be the test set and the remaining $80\%$ constitute the training set.

Creating the 3d-dataset completely goes the same way. The noise, the maximal side-lengths as well as the probabilities
for removing nodes and edges were chosen identically. In order to obtain roughly the same number of nodes per graph as in the 2d case,
we chose $N_x=N_y=N_z=5$, which leads to 125 nodes per graph.
The only difference between the 2d- and the 3d-dataset lies in the number of graphs per Bravais class. We were aiming
for the same total number of graphs (50000), but, in contrast to the 2d case, we now have 14 Bravais classes.
This leads to 3500 graphs per Bravais class.