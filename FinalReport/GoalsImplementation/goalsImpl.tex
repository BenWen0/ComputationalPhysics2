\label{sec:goalsImpl}

The project can be split into two parts. The goal of the first part is to build a GNN that is capable of assigning a 2- or 3-dimensional lattice its Bravais class. 
For this instance, different lattices need to be created programmatically. Furthermore, they have to be labeled so that they can serve as training data.
Once we have a good amount of training data at hand, we need to determine the optimal structure of the GNN. This amounts to finding the
right functions $\phi_{upd},\phi_{mes},\phi_{agg}$ in equation~\ref{eq:mess_pass} and to determine which edge-/node-features do best.
In the second part of the project, we will build upon the results in the first part and take a look at more complex structures. The goal is to distinguish between percolating and non-percolating structures.

\subsection{Lattice Creation}
\label{sec:creationLattice}

To be precise, we will not create lattice according to the definition given in sectio~\ref{sec:brav_latt}, 
as this would require the creation of sets with infinite elements. Clearly, a GNN can only deal with finitely many nodes.
Hence, we will only create finite subsets of lattice but, for sake of simplicity, we will still call them lattices.

Let us start with the creation of two-dimensional lattices.
Choose $a,b\in\R_+$, $\phi\in(0,\pi)$ and set
$e_x=(0,a)^T,\,e_y=(b\cos(\phi), b\sin(\phi))^T$ (for visulaization ogf $a,b$ and $\theta$, see figure~\ref{fig:bravais2D}). 
Then, the set $\tilde{\Omega}=\Z e_x\times\Z e_y\subset\R^2$ is a two-dimensional lattice.
As mentioned, we only work with finite subsets of $\Omega$. Hence, we choose $N_x,N_y\in\N$ and set $\Omega=\N_{\leq N_x}e_x\times\N_{\leq N_y}e_y\subset\tilde{\Omega}$.
This lattice now has $N_xN_y$ elements.
Next, we add some noise to the elements in $\Omega$ and restrict their coordinates to a certain range. For this instance, let $\mu,\sigma,s\in\R_+$ and draw random samples from
a normal distribution with mean value $\mu$ and standard deviation $\sigma$. Secondly, we scale these random numbers by the factor $s$ 
and add the scaled noise to all elements in $\Omega$ (component-wise, i.e. add noise to the first coordinate of the elements in $\Omega$ as well as to the second coordinate).
Restricting all coordinates to a certain range is simply done. Let $x_{max}, y_{max}\in\R_+$ and replace each $(x,y)\in\Omega$ with $(x \mod x_{max}, y \mod y_{max})$ 
Next, we want to turn our lattice $\Omega$ into a graph $(V,E)$. Obviously, we can set $V=\Omega$ and what remains is the choice of edges. For this, we take $r\in\R_+$ and set
\begin{equation*}
    E=\{(v, w)\in V\times V: \norm{v-w}<r\},
\end{equation*}
where $\norm{\cdot}$ denotes the standard norm in $R^d$, i.e. nodes that are close together will be connected.
Lastly, we choose $p_n,p_e\in[0,1]$ and randomly delete nodes with probability $p_n$ (and edges with probability $p_e$ respectively).

The creation of three-dimensional lattices is completely analogously. The only difference is that we have to pick three basis vectors $e_x,e_y,e_z$ which have length $a,b,c\in\R_+$ and 
enclose angles $\phi=\angle(e_x, e_y),\,\psi=\angle(e_x,e_z),\,\chi=\angle(e_y,e_z)$. 

Figure~\ref{fig:examplesLattices} shows two different lattices created by the above procedure.

\begin{figure}
    \centering
    \begin{subfigure}[t]{0.45\textwidth}
        \centering
        \includegraphics[width=\textwidth]{Background_Ch1/square2d.png}
        \caption{2d}
    \end{subfigure}
    \hfill
    \begin{subfigure}[t]{0.45\textwidth}
        \centering
        \includegraphics[width=\textwidth]{Background_Ch1/cubic3d.png}
        \caption{3d}
    \end{subfigure}
    \caption{Examples of lattices created by the procedure described in section~\ref{sec:creationLattice}.}
    \label{fig:examplesLattices}
\end{figure}

According to the above procdeure, we are able, to generate training and test datasets for our GNN, both in 2d and in 3d.
We start with a quick describition of the 2d-dataset. 
First, epending on the specific values of $a,b$ and $\theta$, different graphs for different Bravais classes can be generated 
(take again a look at figure~\ref{fig:bravais2D} as a reminder, which values of $a,b,\theta$ result in which Bravais class).
Recall, that in some cases $\theta=\frac{\pi}{2}$ is required. In order to introduce a bit artifical noise, we allow for values
in the range $(0.99\frac{\pi}{2}, 1.01\frac{\pi}{2})$.
Accordingly, when the side lengths are required to fullfill $a=b$, we allow for deviations of $1\%$, 
i.e. such that $0.99\leq\frac{a}{b}\leq1.01$.
Furthermore, we set the minimal side length to be 0.1 and the maximal side length to be 10. 
The parameters determining the above mentioned noise, where chosen to be $\sigma=0.5,\mu=0$. The amplitude of
the noise in x-direction was taken to be $s=0.07a$ (depeneding on the side-length $a$) as well as $s=0.07b$
in y-direction. The probabilities $p_n,p_e$ for randomly droping nodes/edges were taken to be $p_n=p_e=0.01$
According to these rules, we created 10000 graphs per Bravais class. Since there are 5 Bravais classes in total, the
2d-dataset consists of 50000 graphs, $20\%$ were taken to be the test set and the remaining $80\%4 $ constitute the training set.

\comment{3d  Datensatz}

\subsection{Implementing the GNN}
\label{sec:implGNNBravais}
Once we have a sufficient amount of training data at hand, we need to determine the optimal structure of the GNN as well as
a suitable training procedure. 
This amounts to finding the right functions $\phi_{upd},\phi_{mes},\phi_{agg}$ (cf. equation~\ref{eq:mess_pass}) 
and proper number of message passing layers. More than that, one has to decide, which edge-/node-features 
do best, as well as which hyperparameters and loss function to use during training.

As we do not have extensive computational resources available, we were unable to vary all of these parameters.
We fixed the hyperparameters, the loss function as well as the node-/edge features as follows: 
Both in the 2d and in the 3d case the edge ${n,m}$ between nodes $n$ and $m$ 
carries the difference of the positions of nodes $n$ and $m$ as edge feature $e_{n,m}$.
The nodes $n$ and $m$ do not carry any specific node feature. Each node got the number 1 assigned as 
a feature, i.e. there are basically no node features.
For training the GNN, the standard NAdam optimizer with all its standard settings implemented in the PyTorch library was used.
In particular, the standard setting for the learning rate is $\eta=0.002$.
Each training consisted of 30 epochs. The training datasets were split in batches of size 32.
Since we are interested in classification tasks, we one-hot encoded the Bravais classes and used cross entropy loss as our loss function.

Furthermore, in all the following experiments, we fixed $\phi_{agg}$ to be the function that sums up all its inputs.
All other parameters mentioned at the beginning of this section were varied in the following way:
The function $\phi_{mes}$ was implemented as a general feed forward neural network (cf. section~\ref{sec:intro_nn}) with 
depth $d_m$ and uniform width $w_m$. By depth, we mean the number of hidden layers and by uniform width we mean the number of nodes
in each hidden layer, which was chosen to be uniform over all hidden layers.
Likewise, the function $\phi_{upd}$ was taken to be a general feed forward neural network width depth $d_u$ and uniform width $w_u$.
In principle, as mentioned above, there is a fifth parameter that needs to be optimized, namely the number $d_G$ of message
passing layers. However, five parameters are computationally too expensive to handle. Therefore,
all further experiments were conducted with $d_G=2$. Via grid search, we looked through all possible combinations
of $d_m,w_m,d_u,w_u$ in the following ranges
\begin{align}
    d_m&\in\{1, 2, 3\}, \label{eq:list_settings_start} \\
    w_m&\in\{10, 20, 30\} ,\\
    d_u&\in\{1,2\}, \\
    w_u&\in\{5, 10\}.  
    \label{eq:list_settings_stop} 
\end{align}
In total, this amounts to finding an optimum within 36 parameters. Once we found an optimum on the 2d dataset, we used
these optimal settings and trained on the 3d dataset. 
There are to question we are aiming to answer: First, can we find any correlations between
the widths and depths of our GNN and the training accuracy. 
Second, whether the settings that did best in the 2d case also lead to good results in the 3d case.

A few words on the actual implementation: The implementation is based on PyTorch and PyTorch Geometric (abbreviated PyG), see \cite{PyTorch} and \cite{PyG}. 
Luckily, one does not have to implement the whole message passing scheme.
Instead, the PyG-package is equipped with a base class called MessagePassing. This class is build in a way such that
the function $\phi_{upd},\phi_{mes},\phi_{agg}$ can be freely implemented and everything else works under the hood. 
Hence, it is sufficient to state how these three functions were implemented. It is not necessary to go into 
detail about the  implementation of the whole message passing scheme.

\subsection{On the Problem of Percolation}
\label{sec:implPercolation}
For classification of graphs into percolating and non-percolating ones, we were asked to experiment with a mixture of the GNN described in the previous section
(which layers are message passing layers) and the so called Top-K Pooling layer.
How exactly this mixture looks like depends on the experiment we did and will be explained once we come the specific experiments.
What follows is a rough overview over the working principles of the Top-K Pooling layer. 
For additional information see~\cite{topKPooling}, where this layer was originally proposed.
The following explanation of the pooling procedure might become more clear with
an illustration at hand. See figure~\ref{fig:topKPooling}.
\begin{figure}[h]
    \centering
    \includegraphics[width=0.95\textwidth]{GoalsImplementation/topKpooling}
    \caption{Illustration of the Top-K Pooling layer. See section~\ref{sec:implPercolation} for an explanation. Taken from~\cite{topKPooling}.}
    \label{fig:topKPooling}
\end{figure}
Suppose $G=(V,E)$ is an attributed graph. Recall from section~\ref{sec:fund_gnns}, 
that each node $x\in V$ has a node feature $v_x\in\R^n$. We can organize all node
features in a matrix $X^l\in\text{Mat}(|V|\times n,\R)$ given by
\begin{equation}
    \left(X^l\right)_{x,j}=\left(v_x\right)_j, \quad x\in V, 1\leq j \leq n.
    \label{eq:matNodeFeatures}
\end{equation}
Furthermore, we can define the so-called adjacency matrix $A^l\in\text{Mat}(|V|\times |V|,\R)$ to be
\begin{equation}
    \left(A^l\right)_{x,y}=\begin{cases}
        1 \quad \text{if } (x,y)\in E,\\
        0 \quad \text{if } (x,y)\notin E
    \end{cases}, \quad x,y\in V.
    \label{eq:matAdjacency}
\end{equation}
For $p\in\R^n$, called the projection vector, we define $y=\frac{X^l p}{\norm{p}}$. 
Next, choose $k<|N|$ and select the indices of the $k$ highest entries in $y$ and denote the resulting 
list of indices as $idx\in\N^k$. To each index $i$ in $idx$, there is a corresponding 
node $x_i\in V$. Therefore, we can equivalently think of $idx$ as a subset $\tilde{V}\subset V$. 
These are the nodes that will not be deleted during the pooling process. 
All nodes in $V\setminus\tilde{V}$ will be deleted. 
The deletion is done as follows: Define $\tilde{X}\in\text{Mat}(k\times n,\R)$ to be the matrix
consisting of all node feature $v_x$ for $x\in\tilde{V}$, analogues to equation~\ref{eq:matNodeFeatures}. 
Likewise, define $A^{l+1}\in\text{Mat}(k\times k,\R)$ analogously to equation~\ref{eq:matAdjacency}
to be the adjacency matrix corresponding to the nodes in $\tilde{V}$.
Next, compute the elementwise product of $\tilde{y}$ and $\tilde{X}$ which leads to a new matrix
$X^{l+1}\in\text{Mat}(k\times n,\R)$. This step is called \glqq{}gate operation\grqq{}.
The new node feature matrix $X^{l+1}$ together with the new adjacency matrix $A^{l+1}$
define a new graph $G^{l+1}$ which has $k<|V|$ nodes. Effectively, we have reduced the original graph $G$ with
$|V|$ nodes to a smaller graph. Depending on the choice of the projection vector $p$, we can
achieve different output graphs $G^{l+1}$. Given a specific problem, the goal of the 
Top-K Pooling layer is to learn a suitable projection vector $p$.
The gate operation step ensures, that the projection vector is indeed learnable via standard backpropagation.
The precise argument why the gate operation is necessary for learning $p$ is a bit technical, we refer to~\cite{topKPooling}.
Again, one do not have to implement this layer from scratch, as the PyG library already provides an implementation, where one 
just have to choose the desired $k$.

Obviously, the dataset created in section~\ref{sec:creationLattice} is not appropriate
for the percolation problem. Instead, the dataset for the percolation problem was created by Jonas Buba and his 
colleges. The dataset consists in total of 1614 planar graphs, 822 of which are not percolating, whereas
the remaining 792 are percolating. See figure~\ref{fig:expPercNonPerc} for examples of graphs that are in the dataset.
The whole set was again partitioned in a training set ($80\%$ of the total number of graphs)
and a test set ($20\%$). The nodes have their position in the unit cube as attributes, whereas the edges are attributed with their length.
\begin{figure}[h]
    \centering
    \begin{subfigure}[t]{0.45\textwidth}
        \centering
        \includegraphics[width=\textwidth]{GoalsImplementation/percolating.png}
        \caption{Percolating}
    \end{subfigure}
    \hfill
    \begin{subfigure}[t]{0.45\textwidth}
        \centering
        \includegraphics[width=\textwidth]{GoalsImplementation/nonPercolating.png}
        \caption{Non-Percolating}
    \end{subfigure}
    \caption{Example of graphs that are in the dataset used for the percolation problem 
    (here $r=0.08$, see again section~\ref{sec:intro_percolation} for an interpretation of $r$).}
    \label{fig:expPercNonPerc}
\end{figure}
