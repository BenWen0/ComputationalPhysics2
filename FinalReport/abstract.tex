The project lies at the intersection of Machine Learning and solid-state physics. 
A common task in solid-state physics is the classification of atomic structures, such as those found in crystals.
Machine Learning, on the other hand, is well-known for its ability to handle classification tasks. Combining these two worlds provides a powerful tool for 
classifying different crystal structures. 
As crystal structures are described by the so-called Bravais lattices, which closely relate to graphs, we will need machine learning tools capable of properly handling graph-like data.
Fortunately, in the last few years, a new type of neural networks, exactly designed for this kind of data, emerged: Graph Neural Networks.
The overall goal of this project is to get familiar with GNNs and apply them to classification tasks of Bravais lattices. 
Furthermore, the phenomenon of percolation will be explored.