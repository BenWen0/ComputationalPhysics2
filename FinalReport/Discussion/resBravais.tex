At first, we start by analyzing the results of training on the 2d-dataset and then go on to the 3d case.

\paragraph{Results on the 2d dataset}
As mentioned in section~\ref{sec:implGNNBravais} we tested in total 36 different combinations of values for $d_m,w_m,d_u,m_u$.
Before going into detail which combination led to the best results, we start 
with a more high-level overview:

The average test loss and test accuracy over all 36 combinations can be found in figure~\ref{fig:avgBravais2d}.
\begin{figure}[h]
    \centering
    \includegraphics[width=0.7\textwidth]{Discussion/plots/bravais2dAvg.png}
    \caption{avg bravais 2d}
    \label{fig:avgBravais2d}
\end{figure}
With an accuracy of approximately $90\%$, the problem can be considered solved. 
However, there are non-negligtable difference between models. 
The model that performed best achieved an accuracy about $95\%$, whereas the model that performed worst
only managed to get about $67\%$ accuracy. The training process of both models are dpeicted in figure~\ref{fig:bravais2dBestWorst}.
\begin{figure}[h]
    \centering
    \begin{subfigure}[t]{0.45\textwidth}
        \centering
        \includegraphics[width=\textwidth]{Discussion/plots/bravais2dbest.png}
        \caption{best}
    \end{subfigure}
    \hfill
    \begin{subfigure}[t]{0.45\textwidth}
        \centering
        \includegraphics[width=\textwidth]{Discussion/plots/bravais2dworst.png}
        \caption{worst}
    \end{subfigure}
    \caption{bravais 2d}
    \label{fig:bravais2dBestWorst}    
\end{figure}
Hence, different values for $d_m,w_m,d_u,m_u$ may lead to drastically different outcomes. 

\paragraph{Results on the 3d dataset}
As mentioned in section~\ref{sec:implGNNBravais} we chose the model that 
performed best in the 2d case and tested in on the 3d dataset. 
As explained in the settings $d_m=,w_m,=d_u=,m_u=$ led to the best results 
in the 2d case. Without any further tweaks of these paramters or changes in the training procedure,
the model achieved a test accuracy of $xy\%$ on the 3d dataset (see figure~\ref{fig:bravaisBest3d}).
\begin{figure}[h]
    \centering
    \caption{best bravais 3d}
    \label{fig:bravaisBest3d}
\end{figure}
Out of interest, we trained to worst performing model in the 2d dataset on the 3d dataset too.
Interestingly, it performed equally bad (see figure~\ref{fig:bravaisWorst3d}).
\begin{figure}[h]
    \centering
    \caption{worst bravais 3d}
    \label{fig:bravaisWorst3d}
\end{figure}