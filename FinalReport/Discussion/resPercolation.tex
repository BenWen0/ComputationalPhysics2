As mentioned in~\ref{sec:implPercolation} I was asked to experiment with the TopKPooling layer. 
However, before presenting the results on how well this layer performed, we will give an argument, why the percolation problem 
cannot be solved by a MessagePassing GNN, no matter how clever it might be designed.
Suppose there is a Message Passing GNN, that can solve the percolation problem, i.e. given any graph 
with nodes positioned inside the unit square, it can determine whether or not the graph is percolating. 
Now, take any graph $G$ you like and choose two nodes $n$, $m$, that are not connected by an edge (i.e. $\{n,m\}\notin E$). 
Assign each node different from $n, m$ a position inside 
$(r, 1-r)\times(r, 1-r)$, place node $n$ inside the strip $[0,1]\times [0,r]$ and $m$ inside $[0,1]\times[1-r, 1]$. 
Furthermore, add the new edge $\{n,m\}$, which gives a new graph $\tilde{G}$ that has the same nodes
as $G$ and the same edges plus the one additinal more. Next, run the GNN on the graph $\tilde{G}$. 
Either, the GNN outputs that $\tilde{G}$ is not percolating or it outputs that $\tilde{G}$ is percolating. 
If the graph was percolating, the nodes $n$ and $m$ were already connected in $G$. 
In case $\tilde{G}$ is not percolating, $n$ and $m$ are not connected in $G$. 
In total, we can use our GNN to detect, whether two randomly chosen nodes ($n$ and $m$) in a randomly 
chosen graph ($G$) are connected via a path or not.
However, such a GNN can clearly not exist (suppose there were such a GNN, 
then consider the graph shown in figure~\ref{fig:gnn_connectedComponent}, which leads to
a contradiction).
Hence, a GNN that is capable of solving the percolation problem can not exist too.
\begin{figure}
    \centering
    \begin{tikzpicture}[
        node/.style={circle, inner sep=0pt, fill=black, minimum size=3mm},
        node_missing/.style={circle, inner sep=0pt, fill=black, minimum size=1mm}
    ]
    \node[node, label={$0$}] (n1) [] {1};
    \node[node, label={$1$}] (n2) [right=of n1] {2};
    \node[node, label={$2$}] (n3) [right=of n2] {3};
    \node[node_missing] (nm_1) [right=0.7cm of n3] {};
    \node[node_missing] (nm_2) [right=0.3cm of nm_1] {};
    \node[node_missing] (nm_3) [right=0.3cm of nm_2] {};
    \node[node, label={$2l-1$}] (n4) [right=0.7cm of nm_3] {};
    \node[node, label={$2l$}] (n5) [right=of n4] {};
    \node[node, label={$2l+1$}] (n6) [right=of n5] {};
    
    \draw[very thick]  (n1) -- (n2);
    \draw[very thick]  (n2) -- (n3);
    \draw[very thick]  (n4) -- (n5);
    \draw[very thick]  (n5) -- (n6);

    \end{tikzpicture}
    \caption{No Message Paassing GNN can detect whether two nodes are in the same connected coomponent. 
    Suppose such a GNN exists and that it has $l$-layers. Consider the graph with $2l+2$ nodes depicted above. 
    Since each node can only share information with its $l$-neighrest neighbors, 
    node 0 can only share information with nodes $0,\cdots,l$
    and node $2l+1$ can only receive information from nodes $l+1,\cdots,2l+1$. 
    Hence, there is no possibilty for node $0$ to know about node $2l+1$ and 
    therefore, the GNN cannot detect, whether they are in the same connected component or not.}
    \label{fig:gnn_connectedComponent}
\end{figure}
Besides beeing provable impossible to solve this problem, I was asked to present some training processes nonetheless. 
The resulst can be seen in figure~\ref{fig:trainingPerc}. Unsurprisingly, the GNN was not able to solve the percolation problem.
\begin{figure}
    \centering
    \caption{bla bla bla}
    \label{fig:trainingPerc}
\end{figure}

As we cannot hope to overcome the percolation problem with pure message passing layers, 
a natural idea is to try using layers that do not depend in message passing. 
That is the reason why I was asked work with the TopKPooling-Layer.
However, there is a good reason, why, even with the TopKPolling, we cannot hope to achieve 
anything better than in~\ref{fig:trainingPerc}.
Recall, that the TopKPooling procedure deletes nodes with all its edges and does not create new edges. 
In particular, it does not presever connected components. To illistrate the point a bit more, 
consider the follwoing situation: Suppose there is a percolating graph $G$. 
After going through the TopKPolling layer some node might be deletet, 
so that the resulting graph is not percolating anymore. 
Hence, instead of looking at Pooling layers that do not respect cnnected components, we have to look 
at pooling layers, that preserve connected components.
Depsite beeing a interesting challenge, this goes beyond the scope of this project.
Hence, a TopKPolling layer will not help with the percolatino problem. 
Nonetheless, I was again asked to present some training results. They can be found in figure~\ref{fig:resTopK}. 
As expected, the training does not show any improvements compared to figure~\ref{fig:trainingPerc}.
\begin{figure}
    \centering
    \caption{bla bla bla}
    \label{fig:resTopK}
\end{figure}
 